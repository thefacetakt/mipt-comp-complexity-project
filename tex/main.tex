%------------------------------------------------------------------------------
% Beginning of journal.tex
%------------------------------------------------------------------------------
%
% AMS-LaTeX version 2 sample file for journals, based on amsart.cls.
%
%        ***     DO NOT USE THIS FILE AS A STARTER.      ***
%        ***  USE THE JOURNAL-SPECIFIC *.TEMPLATE FILE.  ***
%
% Replace amsart by the documentclass for the target journal, e.g., tran-l.
%

\documentclass{amsart}

\usepackage[utf8]{inputenc}

\usepackage[english, russian]{babel}


%     If your article includes graphics, uncomment this command.
\usepackage{graphicx}
\usepackage{listings}
\usepackage{algpseudocode}

\newtheorem{theorem}{Theorem}[section]
\newtheorem{lemma}[theorem]{Lemma}

\theoremstyle{definition}
\newtheorem{definition}[theorem]{Definition}
\newtheorem{example}[theorem]{Example}
\newtheorem{xca}[theorem]{Exercise}

\theoremstyle{remark}
\newtheorem{remark}[theorem]{Remark}

\numberwithin{equation}{section}

%    Absolute value notation
\newcommand{\abs}[1]{\lvert#1\rvert}

%    Blank box placeholder for figures (to avoid requiring any
%    particular graphics capabilities for printing this document).
\newcommand{\blankbox}[2]{%
  \parbox{\columnwidth}{\centering
%    Set fboxsep to 0 so that the actual size of the box will match the
%    given measurements more closely.
    \setlength{\fboxsep}{0pt}%
    \fbox{\raisebox{0pt}[#2]{\hspace{#1}}}%
  }%
}


\begin{document}

\title{Выполнимость 3-КНФ}

%    Information for first author
\author{Каргальцев Степан}
%    Address of record for the research reported here
\address{МФТИ, 494}
%    Current address
\email{stepikmvk@gmail.com}

\date{Декабрь 2016}

\keywords{Differential geometry, algebraic geometry}

\maketitle




\section{Постановка задачи}
Построить и имплементировать алгоритм, про который можно доказать следующее:
\begin{itemize}
\item Он распознает выполнимость 3-КНФ;
\item В случае P = NP он делает это за полиномиальное время
\end{itemize}

\section{Решение}

\subsection{Основное решение}

\subsubsection{Описание решения}
Мы предполагаем, что P=NP. Тогда существует машина Тьюринга,
которая за полиномиальное время по выполнимой формуле ищет выполняющий набор.

Запустим следующий псевдокод:

\begin{algorithmic}
\If {$i\geq maxval$}
    \State $i\gets 0$
\Else
    \If {$i+k\leq maxval$}
        \State $i\gets i+k$
    \EndIf
\EndIf
\end{algorithmic}

\algblock[Name]{Start}{End}
\algblockdefx[NAME]{START}{END}%
    [2][Unknown]{Start #1(#2)}%
    {Ending}
\algblockdefx[NAME]{}{OTHEREND}%
    [1]{Until (#1)}
\begin{algorithmic}
\Start
    \Start
        \START[One]{x}
        \END
        \START{0}
        \OTHEREND{\texttt{True}}
    \End
    \Start
    \End
\End
\end{algorithmic}

% \begin{algorithmic}
%     Вход: 3-кнф формула phi
%     Выход:
%     def check_3sat(phi):
%         if is_not_3cnf(phi):
%             return False
%         for N in 1..infty
%             for M in 1..infty
%                 Запустить машину Тьюринга M на N шагов с входом phi
%                 Если получили выполняющий набор то True
% \end{algorithmic}

\subsubsection{Структура кода}

\subsubsection{Тестирование основного решения}

\subsection{Дополнительные решения}

\subsubsection{Тестирование дополнительного решения 1}






\bibliographystyle{amsplain}
\begin{thebibliography}{10}

\bibitem  https://en.wikipedia.org/wiki/P\_versus\_NP\_problem

\bibitem {B} R. Brown, \textit{On a conjecture of Dirichlet},
Amer. Math. Soc., Providence, RI, 1993.

\bibitem {D} R. A. DeVore, \textit{Approximation of functions},
Proc. Sympos. Appl. Math., vol. 36,
Amer. Math. Soc., Providence, RI, 1986, pp. 34--56.

\end{thebibliography}

\end{document}

%------------------------------------------------------------------------------
% End of journal.tex
%------------------------------------------------------------------------------
