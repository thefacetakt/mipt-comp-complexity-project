\documentclass[paper=a4, fontsize=11pt]{scrartcl}
\usepackage[letterpaper, top=0.4in]{geometry}


\usepackage[utf8]{inputenc}
\usepackage[english, russian]{babel}
\usepackage{cmap}
\usepackage{enumitem}
\usepackage[export]{adjustbox}
\usepackage{flafter}
\usepackage{lscape}

\usepackage{algorithm2e}
\usepackage{hyperref}
\SetKwInput{KwData}{Исходные параметры}
\SetKwInput{KwResult}{Результат}
\SetKwInput{KwIn}{Входные данные}
\SetKwInput{KwOut}{Выходные данные}
\SetKwIF{If}{ElseIf}{Else}{если}{тогда}{иначе если}{иначе}{конец условия}
\SetKwFor{While}{до тех пор, пока}{выполнять}{конец цикла}
\SetKw{KwTo}{от}
\SetKw{KwRet}{возвратить}
\SetKw{Return}{возвратить}
\SetKwBlock{Begin}{начало блока}{конец блока}
\SetKwSwitch{Switch}{Case}{Other}{Проверить значение}{и выполнить}{вариант}{в противном случае}{конец варианта}{конец проверки значений}
\SetKwFor{For}{цикл}{выполнять}{конец цикла}
\SetKwFor{ForEach}{для каждого}{выполнять}{конец цикла}
\SetKwRepeat{Repeat}{повторять}{до тех пор, пока}


\usepackage{listings}
\usepackage{algpseudocode}


\parindent=0cm
\setcounter{tocdepth}{4}
\begin{document}


\begin{center}
\textbf{Выполнимость 3-КНФ}
\end{center}

\begin{flushright}
Каргальцев Степан

МФТИ, 494

2016

stepikmvk@gmail.com
\end{flushright}

\tableofcontents

\newpage

\section{Постановка задачи}
Построить и имплементировать алгоритм, про который можно доказать следующее:
\begin{itemize}
\item Он распознает выполнимость 3-КНФ;
\item В случае P = NP он делает это за полиномиальное время
\end{itemize}

\section{Решение}


\subsection{Описание решения}
\bigskip

Псевдокод:\\


\begin{algorithm}[H]
		\SetAlgoLined
		\KwData{Выполнимая формула $\varphi$}
		\KwResult{Выполняющий набор для $\varphi$}
		\For{$n \leftarrow 1 \ldots \infty$}{
			\For{$m \leftarrow 1 \ldots n$}{
                Запустить машину номер $m$ на $n$ шагов на входе $\varphi$;
                \If{Машина $m$ завершилась} {
                    \If{Выход машины $m$ --- выполняющий набор для $\varphi$} {
                        Вернуть выход машины $m$;
                    }
                }
            }
		}
\end{algorithm}


Покажем, что данный код работает за полиномиальное время. Действительно,
если P=NP, то существует машина тьюринга $M$ такая, что она по выполнимой
формуле $\varphi$ выдает выполняющий набор для этой формулы за полиномиальное
время. (Вообще, P=NP напрямую означает лишь существование машины Тьюринга,
которая распознает принадлежность $\varphi$ к $3-SAT$, но в [1] в разделе 3.5
доказывается, что
в случае P=NP можно также быстро решать соответствующую задачу поиска, чем мы и
пользуемся).\\


Пусть эта машина работает не более, чем за $P(|x|)$ шагов на входе $x$, где $P$
--- некоторый полином, а ее номер в нашей нумерации $M$
(подробнее про реализацию нумерации и перебора машин ---
 в разделе \textit{Структура кода}).\\


Положим $T := \max(M, P(|\varphi|))$.\\


Тогда мы сделаем не более чем
$1^2 + 2^2 + \ldots + T^2 = Q(T)$ шагов (под шагами подразумевается шаги
эмулируемых машин тьюринга),
($Q(x) = \frac{x(x + 1)(2x + 1)}{6}$ --- фиксированный полином). Действительно,
прежде чем переменая во внешнем цикле достигнет значения $P(|\varphi|)$ пройдет
не более $Q(P(|\varphi|))$ шагов, а прежде чем переменная внутреннего цикла
достигнет значения $M$ пройдет не более $Q(M)$ шагов.\\


Когда внешняя переменная станет равна $P(|\varphi|)$, а внутренняя $M$, то мы
запустим машину $M$ на $P(|\varphi|)$ шагов и получим выполняющий набор
(однако, возмножно, мы получили его раньше и вышли).\\


Посмотрим, на что мы тратим время:\\


\begin{enumerate}
\item Эмуляция машин Тьюринга
\item Проверка корректности выполнимого набора
\item Итерации по циклам
\end{enumerate}


Эмуляцию машин Тьюринга мы будем делать с ухудшением времени в константное число
раз. Итераций по циклам никак не больше чем шагов машин Тьюринга, проверка
корректности выполнимого набора выполняется за полиномиальное от размера формулы
время и таких проверок будет не больше, чем итераций по циклам. Итого наше время
работы можно ограничить следующей величиной:\\


$$(P_1(|\varphi|)  + C)\cdot Q(T) =
    P_1(|\varphi|) + C)\cdot Q(\max(M, P(|\varphi))) \leq$$

$$P_1(|\varphi|) + C)\cdot Q(M) + P_1(|\varphi|) + C)\cdot Q(P(|\varphi))$$,


где $P_1$ --- полином, ограничивающий время проверки корректности выполняющего
набора, а $C$ --- константа ухудшения при эмуляции машины Тьюринга.
В любом случае, учитывая, что $M$ -- это константа, а композиция, произведение,
сумма полиномов --- полином, получаем, что время работы
алгоритма полиномиальное.\\


(Доказательством утверждений про полиномиальность проверки корректности и
константного ухудшения эмуляций машин Тьюринга будет является код, решающий
поставленные задачи за заявленное время)


\subsection{Проблема и ее решение}


Внимательный читатель заметит, что мы научились находить выполняющий набор
(и доказывать выполнимость) выполнимых формул за полиномиальное время (что,
безусловно, радует), но ничего не сделали с невыполнимыми формулами. Моих
интелектуальных способностей хватило на три следующих выхода из ситуации:


\subsubsection{Проигнорировать}


Давайте интерпретировать условие ("Распознать выполнимость 3-КНФ за
полиномиальное время") как "Понимать (доказывать), что строка является
выполнимой 3-КНФ за полиномиальное время". В таком случае мы решили задачу.

\subsubsection{Поперебирать доказательства}


Давайте считать, что известно, что задача решаемая (нам же не дадут нерешаемых
задач как семестровый проект, правда?). Тогда можно сделать следующее:
перебирать машины тьюринга, и параллельно перебирать
доказательства утверждений "Машина Тьюринга $M_i$ решает 3-SAT за полиномиальное
время в случае P=NP" (Заметим, что нельзя перебирать доказательства утверждений
вида  "Машина Тьюринга $M_i$ решает 3-SAT за полиномиальное время",
потому что из того, что P=NP еще не следует доказуемость этого утверждения).\\


Как только мы нашли машину Тьюринга, про которую существует доказательство
полиномиальности ее работы, запустим ее на нашем входе и победим.\\

Повторюсь, что доказуемость перебираемого утверждения я утверждаю в
предположении решаемости данной мне задачи. Поэтому, в частности, это решение
плохо (доказывать что-то тем, что раз мне это дали как упражнение, то это
решаемое не самое математичное рассуждение). Поэтому я упомяну это решение как
забавное, а сам попытаюсь перейти к чему-нибудь еще.

\subsubsection{Сделать хоть что-то с плохом случае}

Схоже с вариантом <<Проигнорировать>>, но в отличие от него а) решает задачу
в формулировке <<Понять, принадлежит ли $\varphi$ 3-SAT>> и б) добавляет данной
работе нетривиальности. \\

В чем идея --- давайте модифицируем основной алгоритм. Во-первых, при входе
проверим, что $\varphi$ --- 3-КНФ формула. Во-вторых, по происшествии некоторого
количества шагов (или просто параллельно) запустим один из
существующих 3-SAT-решателей. Рассмотрим 2 основных примера.\\


\paragraph{Наивный решатель}

Перебираем все возможные наборы формул, проверяем, является ли текущий набор
выполняющим. Если является --- формула выполнима, вот сертификат. Если никакой
набор не является выполняющим --- формула невыполнима. Работает, очевидно,
за $\mathcal{O}(N2^k)$, где $N$ --- длина формулы, $k$ ---
количество переменных\\


\paragraph{DPLL (Davis-Putnam-Logemann-Loveland)}
Вторая глава [2] утверждает, что большинство лучших SAT-решателей (по крайней
мере в 2008 году) были основаны на процедуре DPLL. Это процедура, по сути,
является перебором с отсечениями. \\

Тут применяются отсечения трех типов:

1) Вначале оценить литералы, входящие без
своих отрицаний (например, если в $\varphi$ входят только $\neg p$, а просто
$p$ не встречается) истинной (т.е если было только $p$ то положит $p=True$, а
если было только $\neg p$, то $p=False$).

2) В процессе перебора, если остался дизъюнкт, в котором ровно один литерал
не оценен, то оценить его, если все остальные --- нули (чтобы этот дизъюнкт
стал истинным), и выкинуть этот дизъюнкт из рассмотрения.

3) Если мы нашли выполняющий набор, надо остановиться.

Легко видеть, что все три отсечения являются корректными.

Псевдокод:\\

\begin{algorithm}[H]
	\SetAlgoLined
		\Function{UnitPropagate}{} {

		\KwData{$\varphi$ - булева формула в КНФ, $\rho$ --
				некоторая оценка переменных (возможно, не всех) $\varphi$}
		\KwResult{Либо эквивалентная $\varphi$ формула, не содержащая
				  дизъюнктов из одной значимой переменной, либо индикатор,
				  что $\varphi$ тождественно ложна при такой оценке переменных}

        \While{В $\varphi$ есть дизъюнкт, в котором ровно один неоцененный
               литерал $\alpha$ И не найдено противоречие} {
			   \If{Если в этом дизъюнкте все, литералы кроме $\alpha$ оценены нулем} {
	               \eIf{$\alpha = p$} {
	                    $\rho$[p] $\leftarrow$ True
	               } {
	                    $\rho$[p] $\leftarrow$ False
	               }
			   }
			   Выкинуть этот дизъюнкт.
        }
        \If{Найдено противоречие} {
            Вернуть '$\varphi$ --- ложна'
        }
        Вернуть $\rho$
		}

\end{algorithm}


\begin{algorithm}[H]
		\Function{DPLL}{} {

    	\KwData{$\varphi$ - булева формула в КНФ, $\rho$ --
                некоторая оценка переменных (возможно, не всех) $\varphi$}
    	\KwResult{Либо выполняющий набор для $\varphi$, либо индикатор,
                  что $\varphi$ тождественно ложна при такой оценке переменных}
        $\rho \leftarrow$ UnitPropagate($\varphi$, $\rho$);

        \If{$\rho$ = '$\varphi$ --- ложна'} {
            Вернуть '$\varphi$ --- ложна'
        }

    	\If{Все переменные оценены} {
            \eIf{$\varphi$ истинна на $\rho$} {
                Вернуть $\rho$
            } {
                Вернуть '$\varphi$ --- ложна'
            }
        }
        Выбрать переменную $x$;

        $\rho$[$x$] $\leftarrow$ True;

        result $\leftarrow$ DPLL($\varphi$, $\rho$);

        \If{result != '$\varphi$ --- ложна'}{
            Вернуть result;
        }

        $\rho$[$x$] $\leftarrow$ False;

        Вернуть DPLL($\varphi$, $\rho$)
		}
\end{algorithm}

\pagebreak

\begin{algorithm}[H]
	\Function{Check3SAT}{} {

        \KwData{$\varphi$ - булева 3-КНФ формула}
        \KwResult{Либо выполняющий набор для $\varphi$, либо индикатор,
                  что $\varphi$ тождественно ложна.}

		pre\_rho $\leftarrow \{\}$;

		\While{в $\varphi$ есть литерал $\alpha$, отрицание
               которого не входит в $\varphi$}{

               \eIf{$\alpha = p$}{
                    pre\_rho[p] $\leftarrow$ True
               }{
                    pre\_rho[p] $\leftarrow$ False
               }
               Выкинуть из $\varphi$ все дизъюнкты, в которые входит $\alpha$.
        }
        DPLL\_Result  $ \leftarrow DPLL(\varphi, \{\})$;

        \eIf {DPLL\_Result = \textit{$\varphi$ тождественно ложна}}{
            Вернуть DPLL\_Result;
        }{
            Вернуть DPLL\_Result $\cup$ pre\_rho
		}
	}
\end{algorithm}



\subsection{Структура кода}

Пришло время вспомнить, что алгоритмы надо не только описывать словестно, но
и реализовывать.\\

Все реализовано на языке python3,
код лежит по адресу \url{https://gitlab.com/thefacetakt/mipt-comp-complexity-project/tree/master/code} \\

Для начала расскажем про код основного алгоритма:\\

\subsubsection{tm.py}

В этом файле содержится описание реализации машины Тьюринга и вспомогательных
вещей (таких как двусторонняя бесконечная лента). Машина тьюринга, как водится,
задается функцией перехода (delta) (ну и количеством состояний).
Алфавит будет везде использоваться $\Sigma = ['0', '1', '\&', '\|', '\sim']$ и
$\Gamma = \Sigma \cup \{'\#'\}$. Стартовое состояние всегда 0, конечное всегда
$n - 1$. Из курса логики известна эквивалентность такой машины Тьюринга
обыкновенной). \\

Заметим, что функция $make\_step$ не содержит циклов и выполняется за константное
число операций, как и было завлено выше. (Строго говоря, внутри $make\_step$
присутствует вызов функции $\_\_getitem\_\_$ класса $TwoSidedTape$, в которой
может происходить расширение массива на несколько ячеек, что не назовешь
константой. Однако, поскольку действие машины Тьюринга локально [то есть если
мы трогаем ячейку $i$ то на предыдущем шаге мы трогали либо ячейку
$i - 1$, либо $i + 1$] подобных линейных спецжффектов не возникает).\\

\subsubsection{tm\_iterating.py}

В этом файле реализована функция последовательного перебора машин Тьюринга.
Как мы из перебираем: сначала перебираем число состояний (generate\_all\_tms), а
внутри перебираем все машины тьюринга с данным числом состояний на нашем
алфавите (то есть по сути, перебираем функцию перехода).\\

Из неочевидных моментов здесь может быть конструкция yield --- она позволяет
генерировать последовательность машин Тьюринга "лениво", подавая их одна за
одной по требованию (приостанавливая перебор). Это очень удобно в нашем случае,
ведь перебирать машин нам надо --- бесконечно много.
Более подробно про конструкцию yield и концепцию генераторов можно прочитать
в официальной документации [3] \\


Теперь про вспомогательные функции:\\

\subsubsection{cnf\_utils.py}

В этом файле реализованы 2 функции: $check\_3cnf$ и $run\_3cnf$.
Первая проверяет,
что строчка является корректной 3-КНФ формулой
(считая, что все переменные -- это числа в двоичной системе от 0 до $n - 1$, \&
отвечает за конъюнкцию, $|$ --- за дизъюнкцию, $\sim$ --- за отрицание.
Приоритеты --- стандартные), и в случае успеха разбивает формулу на дизъюнкты,
а дизъюнкты на литералы.\\

Функция $run\_3cnf$ производит вычисление значения формулы (в разобраном виде) на
данной оценке.\\

Из кода легко видеть, что обе функции работают за полиномиальное время.

\subsubsection{SAT-солверы}

\paragraph{Наивное решение}

Наивное решение лежит в файле \texttt{solutions/stupid.py} и делает ровно то,
что и было описано выше

\paragraph{DPLL}

DPLL алгоритм, как несложно догадаться, лежит в файле
\texttt{solutions/dpll.py} и является переводом на питон псевдокода, описанного
выше.

\subsubsection{main.py}

В этой файле и реализован основной алгоритм, прямо как описано в самом начале
работы


\subsection{Тестирование}

Поскольку алгоритм работает с очень очень большой констаной, основным
инструментом будет модульное тестирование. (Мы будем проверять на корректность
программу по частям). Все тесты собраны в папке \texttt{tests}.
Тесты запускаются из папки \texttt{code} командами
\texttt{python3 -m tests.<название теста>} или в папке \textttt{coverage}
есть скрипт \texttt{test.sh} (для его работы нужна поставленная бибиотека
coverage)  Можно посмотреть покрытие тестами
(\texttt{coverage/htmlcov/index.html})

Наименование устроено очень просто -- файл \texttt{tests/<name.py>} тестирует
содержимое файла \texttt{name.py} (За исключением файла \texttt{tests/solutions.py},
он тестирует файлы в папке \texttt{solutions})

\section{Список литературы}

\begin{enumerate}
\item Д. В. Мусатов, \textit{Сложность вычислений. Конспект лекций}
МФТИ, 2016

\item F. van Harmelen, V. Lifshitz, B.Potter
    \textit{Handbook of Knowledge Representation}, 2008

\item https://wiki.python.org/moin/Generators

\item https://en.wikipedia.org/wiki/P\_versus\_NP\_problem

\end{enumerate}

\end{document}
