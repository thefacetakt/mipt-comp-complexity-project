%------------------------------------------------------------------------------
% Beginning of journal.tex
%------------------------------------------------------------------------------
%
% AMS-LaTeX version 2 sample file for journals, based on amsart.cls.
%
%        ***     DO NOT USE THIS FILE AS A STARTER.      ***
%        ***  USE THE JOURNAL-SPECIFIC *.TEMPLATE FILE.  ***
%
% Replace amsart by the documentclass for the target journal, e.g., tran-l.
%

\documentclass{amsart}

\usepackage[utf8]{inputenc}
\usepackage[english, russian]{babel}


% Перевод плагина
 \usepackage{algorithm2e}
\SetKwInput{KwData}{Исходные параметры}
\SetKwInput{KwResult}{Результат}
\SetKwInput{KwIn}{Входные данные}
\SetKwInput{KwOut}{Выходные данные}
\SetKwIF{If}{ElseIf}{Else}{если}{тогда}{иначе если}{иначе}{конец условия}
\SetKwFor{While}{до тех пор, пока}{выполнять}{конец цикла}
\SetKw{KwTo}{от}
\SetKw{KwRet}{возвратить}
\SetKw{Return}{возвратить}
\SetKwBlock{Begin}{начало блока}{конец блока}
\SetKwSwitch{Switch}{Case}{Other}{Проверить значение}{и выполнить}{вариант}{в противном случае}{конец варианта}{конец проверки значений}
\SetKwFor{For}{цикл}{выполнять}{конец цикла}
\SetKwFor{ForEach}{для каждого}{выполнять}{конец цикла}
\SetKwRepeat{Repeat}{повторять}{до тех пор, пока}

%     If your article includes graphics, uncomment this command.
\usepackage{graphicx}
\usepackage{listings}
\usepackage{algpseudocode}

\newtheorem{theorem}{Theorem}[section]
\newtheorem{lemma}[theorem]{Lemma}

\theoremstyle{definition}
\newtheorem{definition}[theorem]{Definition}
\newtheorem{example}[theorem]{Example}
\newtheorem{xca}[theorem]{Exercise}

\theoremstyle{remark}
\newtheorem{remark}[theorem]{Remark}

\numberwithin{equation}{section}


\parindent=0cm

\begin{document}

\title{}

%    Information for first author
\author{Каргальцев Степан}
%    Address of record for the research reported here
\address{МФТИ, 494}
%    Current address
\email{stepikmvk@gmail.com}

\date{Декабрь 2016}

\begin{center}
\textbf{Выполнимость 3-КНФ}
\end{center}

\maketitle




\section{Постановка задачи}
Построить и имплементировать алгоритм, про который можно доказать следующее:
\begin{itemize}
\item Он распознает выполнимость 3-КНФ;
\item В случае P = NP он делает это за полиномиальное время
\end{itemize}

\section{Решение}


\subsection{Описание решения}
\bigskip

Псевдокод:\\


\begin{algorithm}[H]
		\SetAlgoLined
		\KwData{Выполнимая формула $\varphi$}
		\KwResult{Выполняющий набор для $\varphi$}
		\For{$n \leftarrow 1 \ldots \infty$}{
			\For{$m \leftarrow 1 \ldots n$}{
                Запустить машину номер $m$ на $n$ шагов на входе $\varphi$;
                \If{Машина $m$ завершилась} {
                    \If{Выход машины $m$ --- выполняющий набор для $\varphi$} {
                        Вернуть выход машины $m$;
                    }
                }
            }
		}
	}
\end{algorithm}


Покажем, что данный код работает за полиномиальное время. Действительно,
если P=NP, то существует машина тьюринга $M$ такая, что она по выполнимой
формуле $\varphi$ выдает выполняющий набор для этой формулы за полиномиальное
время. (Вообще, P=NP напрямую означает лишь существование машины Тьюринга,
которая распознает принадлежность $\varphi$ к $3-SAT$, но в [1] в разделе 3.5
доказывается, что
в случае P=NP можно также быстро решать соответствующую задачу поиска, чем мы и
пользуемся).\\


Пусть эта машина работает не более, чем за $P(|x|)$ шагов на входе $x$, где $P$
--- некоторый полином, а ее номер в нашей нумерации $M$
(подробнее про реализацию нумерации и перебора машин ---
 в разделе \textit{Структура кода}).\\


Положим $T := \max(M, P(|\varphi|))$.\\


Тогда мы сделаем не более чем
$1^2 + 2^2 + \ldots + T^2 = Q(T)$ шагов (под шагами подразумевается шаги
эмулируемых машин тьюринга),
($Q(x) = \frac{x(x + 1)(2x + 1)}{6}$ --- фиксированный полином). Действительно,
прежде чем переменая во внешнем цикле достигнет значения $P(|\varphi|)$ пройдет
не более $Q(P(|\varphi|))$ шагов, а прежде чем переменная внутреннего цикла
достигнет значения $M$ пройдет не более $Q(M)$ шагов.\\


Когда внешняя переменная станет равна $P(|\varphi|)$, а внутренняя $M$, то мы
запустим машину $M$ на $P(|\varphi|)$ шагов и получим выполняющий набор
(однако, возмножно, мы получили его раньше и вышли).\\


Посмотрим, на что мы тратим время:\\


\begin{enumerate}
\item Эмуляция машин Тьюринга
\item Проверка корректности выполнимого набора
\item Итерации по циклам
\end{enumerate}\\


Эмуляцию машин Тьюринга мы будем делать с ухудшением времени в константное число
раз. Итераций по циклам никак не больше чем шагов машин Тьюринга, проверка
корректности выполнимого набора выполняется за полиномиальное от размера формулы
время и таких проверок будет не больше, чем итераций по циклам. Итого наше время
работы можно ограничить следующей величиной:\\


$$(P_1(|\varphi|)  + C)\cdot Q(T) =
    P_1(|\varphi|) + C)\cdot Q(\max(M, P(|\varphi))) \leq$$

$$P_1(|\varphi|) + C)\cdot Q(M) + P_1(|\varphi|) + C)\cdot Q(P(|\varphi))$$,


где $P_1$ --- полином, ограничивающий время проверки корректности выполняющего
набора, а $C$ --- константа ухудшения при эмуляции машины Тьюринга.
В любом случае, учитывая, что $M$ -- это константа, а композиция, произведение,
сумма полиномов --- полином, получаем, что время работы
алгоритма полиномиальное.\\


(Доказательством утверждений про полиномиальность проверки корректности и
константного ухудшения эмуляций машин Тьюринга будет является код, решающий
поставленные задачи за заявленное время)


\subsection{Проблемы}\\


Внимательный читатель заметит, что мы научились находить выполняющий набор
(и доказывать выполнимость) выполнимых формул за полиномиальное время (что,
безусловно, радует), но ничего не сделали с невыполнимыми формулами. Моих
интелектуальных способностей хватило на три следующих выхода из ситуации:


\subsubsection{Проигнорировать}\\


Давайте интерпретировать условие ("Распознать выполнимость 3-КНФ за
полиномиальное время") как "Понимать (доказывать), что строка является
выполнимой 3-КНФ за полиномиальное время". В таком случае мы решили задачу.

\subsubsection{Поперебирать доказательства}\\


Давайте считать, что известно, что задача решаемая (нам же не дадут нерешаемых
задач как семестровый проект, правда?). Тогда можно сделать следующее:
перебирать машины тьюринга, и параллельно перебирать
доказательства утверждений "Машина Тьюринга M_i решает 3-SAT за полиномиальное
время в случае P=NP" (Заметим, что нельзя перебирать доказательства утверждений
вида  "Машина Тьюринга M_i решает 3-SAT за полиномиальное время",
потому что из того, что P=NP еще не следует доказуемость этого утверждения).
\\


Как только мы нашли машину Тьюринга, про которую существует доказательство
полиномиальности ее работы, запустим ее на нашем входе и победим.

\\

Повторюсь, что доказуемость перебираемого утверждения я утверждаю в
предположении решаемости данной мне задачи. Поэтому, в частности, это решение
плохо (доказывать что-то тем, что раз мне это дали как упражнение, то это
решаемое не самое математичное рассуждение). Поэтому я упомяну это решение как
забавное, а сам попытаюсь перейти к чему-нибудь еще.

\subsubsection{Сделать хоть что-то с плохом случае}

Схоже с вариантом "Проигнорировать", но в отличие от него а) решает задачу
в формулировке "Понять, принадлежит ли $\varphi$ 3-SAT" и б) добавляет данной
работе нетривиальности. \\

В чем идея --- давайте модифицируем основной алгоритм. Во-первых, при входе
проверим, что $\varphi$ --- 3-КНФ формула. Во-вторых, по происшествии некоторого
количества шагов (или просто параллельно) запустим один из
существующих 3-SAT-решателей.

\subsection{Структура и описание кода}

\subsubsection{Структура кода}

\subsubsection{Тестирование основного решения}

\subsection{Дополнительные решения}

\subsubsection{Тестирование дополнительного решения 1}






\bibliographystyle{amsplain}
\begin{thebibliography}{10}

\bibitem [1] А Д. В. Мусатов, \textit{Сложность вычислений. Конспект лекций}
МФТИ, 2016

\bibitem [2] Б https://en.wikipedia.org/wiki/P\_versus\_NP\_problem

\bibitem [3] В Brown, \textit{On a conjecture of Dirichlet},
Amer. Math. Soc., Providence, RI, 1993.

\end{thebibliography}

\end{document}

%------------------------------------------------------------------------------
% End of journal.tex
%------------------------------------------------------------------------------
